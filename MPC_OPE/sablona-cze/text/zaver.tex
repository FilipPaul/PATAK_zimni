\chapter*{Závěr}
\phantomsection
\addcontentsline{toc}{chapter}{Závěr}

Výsledkem odborné praxe je realizace univerzálního programovacího pracoviště.
Kromě své hardwarové části je k pracovišti vytvořena softwarová platforma, která
umožňuje relativně jednoduchou úpravu a rozšiřování funkčnosti.\\

K pracovišti byla vytvořena kompletní dokumentace k elektrické, mechanické a softwarové části.
Přičemž práce studenta byla zaměřena převážně na elektrickou a softwarovou část.
Dále byla vytvořena analýza rizik, předpis údržby a další předávací protokoly.\\

V současné době je pracoviště ve výrobním provozu firmy SIEMENS a postupně se 
rozšiřuje o další funkce a adaptéry.
Během dosavadního provozu bylo odhaleno několik chyb a nedostatků, které se postupně odstraňují.
Hlavním přínosem práce (kromě finanční odměny) je vytvoření softwarové platformy pro jednoduché propojování
různých zařízení.\\

\section*{Poděkování}
Závěrem bych chtěl poděkovat odbornému vedoucímu práce Radomilu Havlínovi. Jsem velmi rád,
že mi umožnil vést tento 3 týdenní až roční projekt.
Jeho revírem jsou revize,
Jeho tempo bylo vražedné, našimi 
protivníky byly korporátní schůzky, nefunkční hardware a nedostatek času.
Byl v nasazení ve dne, v noci a jeho úkolem je zajistit bezpečnost. Děkuji.

\vfill
\begin{flushright}
\begin{tabular}{@{}p{2.5in}@{}}
\hrulefill \\
Vedoucí práce \\
Ing. Radomil Havlín
\end{tabular}
\end{flushright}

